\documentclass[a4paper,12pt]{article}

\usepackage[dvipdfm]{graphicx}
\usepackage[russian]{babel}
\usepackage{fontspec}
\usepackage[left=20mm,top=25mm,right=15mm,bottom=15mm,headsep=6pt,headheight=22pt]{geometry}
\usepackage{multirow}
\usepackage{longtable}
\usepackage{tabu}
\usepackage{caption}
\usepackage{setspace}
\usepackage{fancyhdr}
\usepackage[at]{easylist}

\pagestyle{fancy}

\setmainfont{Times New Roman}
\setmonofont{Go Mono}

\setlength{\parindent}{1.25cm}
\renewcommand{\baselinestretch}{1.5}
\renewcommand{\smallskip}{\vskip{6pt}}
\renewcommand{\medskip}{\vskip{6pt}}
\renewcommand{\bigskip}{\vskip{12pt}}
%\setlength{\skip}{0}
\frenchspacing

\def\nohyphen{
	\pretolerance=10000
	\tolerance=2000 
	\emergencystretch=10pt
}

% Списки

%% Список с буллетами
\def\bullist{
	\ListProperties(
	%Hide=100,
	Hang=true,
	Progressive*=.65cm,
	Margin=1.25cm,
	Align=fixed,
	%iq 3000 move: скрываем номера \фантом'ом, и вместо них пихаем буллеты
	Style*={\scshape$\bullet$\phantom},
	% АХАХАХАХ СУКА ЭТО СРАБОТАЛО
	% ну почти.. второй уровень косячит. для демо хватит.
	% TODO уровни после
	)
}

%% Нумерованный список
\def\numlist{
	\ListProperties(
	Hang=true,
	Progressive*=.65cm,
	Margin=1.25cm,
	Align=fixed,
}

% Колонтитулы
% TODO номер страницы должен отступать от основного текста
\newcommand*\sethdr[2] {
	\fancyhf{}
	\fancyhead[C]{
		\fontsize{12pt}{0pt}\selectfont
		\thepage
	}
	\fancyhead[L]{\vspace{12pt}\fontsize{10pt}{0pt}\selectfont\it #1}
	\fancyhead[C]{
		\fontsize{12pt}{0pt}\selectfont
		\thepage
	}
	\fancyhead[R]{\vspace{12pt}\fontsize{10pt}{0pt}\selectfont\it #2}
}

\renewcommand{\thepart}{\arabic{part}}

\newcommand*\unpart[1]{
	% TODO 12пт в начале
	\newpage
	\vspace{6pt}
	%\vspace как всегда ведет себя как долбоёб...
	%\vskip не лучше
	% короче ебаный латех не хочет мне вставлять 12 пунктов отступа сразу после page break
	\noindent
	\centerline{\bf\uppercase{#1}}
	\vskip 12pt
	\addcontentsline{toc}{part}{#1}
}

% TODO мб звёздочка лишняя?
\renewcommand*\part[1]{
	\newpage
	\addtocounter{part}{1}
    	\addcontentsline{toc}{part}{#1}
	\vspace{6pt}
	\noindent
	\centerline{\bf\uppercase{{\thepart}\ #1}}
	\vskip 12pt
}

\renewcommand*\section[1]{
	{
		\nohyphen
		\linespread{1}
		\addtocounter{section}{1}
		\addcontentsline{toc}{section}{#1}
		\vspace{12pt}
		\noindent
		\hspace* {.5cm}
		{\raggedright\bf\thepart.\thesection\ #1}
		\vskip 12pt
	}
}

\renewcommand*\subsection[1]{
	{
		\nohyphen
		\linespread{1}
		\addtocounter{subsection}{1}
		\addcontentsline{toc}{paragraph}{#1}
		\vspace{6pt}
		\noindent
		\hspace* {.75cm}
		{\raggedright\bf\thepart.\thesubsection\ #1}
		\vskip 6pt
	}
}

% Настройка таблиц
\tabulinesep=_6pt
\renewcommand{\thetable}{\thepart.\arabic{table}}
\captionsetup[table]{singlelinecheck=false,justification=raggedright,position=top,%
	format=plain,font={rm,onehalfspacing},labelsep=endash,indention=0pt,%
	skip=6pt,belowskip=0pt
	}
\def\cch{\multicolumn{1}{|c|}}
\def\cct{\multicolumn{1}{c|}}
\def\nr{\\\hline}
\newcommand*\autohead[2]{
	#1\endfirsthead
	\caption*{Продолжение таблицы \ref{#2}}\nr
	#1
	\endhead
}

% Иллюстрации
\DeclareCaptionLabelFormat{myfig}{Рисунок #2}
\captionsetup[figure]{singlelinecheck=false,justification=centering,position=bottom,%
	format=plain,font={rm,onehalfspacing},labelformat=myfig,labelsep=endash,indention=0pt,%
	skip=6pt,belowskip=0pt
	}
\renewcommand{\thefigure}{\thepart.\arabic{figure}}

\addto{\captionsrussian}{
	\renewcommand*{\contentsname}{
		\noindent
		\centerline{\bf\uppercase{Содержание}}
		\vskip 12pt
	}
}

\setlength{\belowcaptionskip}{-10pt}
\setlength{\textfloatsep}{6pt}

\usepackage{framed}

\sethdr{Кафедра АСОИУ}{Требования к оформлению студенческих работ}

\begin{document}
\unpart{Введение}

В настоящее время получение высшего профессионального образования подразумевает
не просто получение некоторого набора знаний, но целого комплекса компетенций – знаний,
умений, навыков, направленных на решение профессиональных задач. Одной из компетенций
является умение оформлять научно-технические отчёты по результатам выполненной работы.
На формирование этой компетенции направлено выполнение курсовых проектов и работ,
отчётов по практике.

Выпускная квалификационная работа является заключительным этапом обучения
студента на соответствующей ступени высшего профессионального образования и имеет своей
целью:
\begin{easylist}\NewList\bullist
	@ систематизацию, закрепление и расширение теоретических и практических знаний по
	соответствующей специальности (направлению подготовки), а также формирование
	навыков применения этих знаний при решении конкретных научных, научно-
	технических, экономических, социально-культурных и производственных задач;
	@ развитие навыков ведения самостоятельной работы и овладения методикой
	теоретических, экспериментальных и научно-практических исследований,
	используемых при выполнении выпускной квалификационной работы;
	@ приобретение опыта систематизации полученных результатов исследований,
	формулировку новых выводов и положений как результатов выполненной работы, а
	также опыта их публичной защиты.
\end{easylist}

Выпускная квалификационная работа (ВКР) в области информатики и информационных
технологий представляет собой законченную разработку в профессиональной области, где:

\begin{easylist} \bullist
	@ сформулирована актуальность и место решаемой задачи информационного обеспечения 
	в предметной области;
	@ анализируется литература и информация, полученная с помощью глобальных сетей
	по функционированию подобных систем в данной области или в смежных
	предметных областях;
	@ определяются и конкретно описываются выбранные выпускником объемы, методы и
	средства решаемой задачи, иллюстрируемые данными и формами выходных
	документов, используемых при реализации поставленной задачи информационного
	обеспечения;
	@ анализируются предлагаемые пути, способы, а также оценивается экономическая,
	техническая и (или) социальная эффективность их внедрения в информационную
	среду в области применения.
\end{easylist}

ВКР бакалавра может быть представлена в виде дипломной работы, либо в виде
дипломного проекта. ВКР магистра представляет собой диссертацию, то есть научную работу,
при её выполнении оформлении следует придерживаться требований ГОСТ 7.32-2001.

Дипломная работа –-- самостоятельная комплексная работа студента, главной целью и
содержанием которой является всесторонний анализ или научные исследования по одному из
вопросов теоретического или практического характера по профилю специализации. Дипломная
работа представляет собой теоретическое или экспериментальное исследование одной из
актуальных научных проблем. Результаты работы оформляются в виде пояснительной записки с
приложением графиков, таблиц, чертежей, схем и других документов.

Дипломный проект –-- это ВКР, которая содержит решение поставленной задачи,
оформленное в виде конструкторских, технологических, программных и других проектных
документов. Главной целью и содержанием дипломного проекта являются разработка
проектного решения, связанного с созданием или совершенствованием информационной
системы на базе использования современных информационных технологий, средств
вычислительной техники и передачи данных, экономико-математических методов и моделей,
разработка технологических процессов обработки информации и решение организационных
вопросов управления производством. Результаты работы оформляются в виде пояснительной
записки с приложением графиков, таблиц, чертежей, схем и других документов, а также
разработанного программного продукта.

Курсовая работа представляет собой небольшое по объёму научное исследование.
Результаты работы оформляются в виде пояснительной записки с приложением графиков,
таблиц, чертежей, схем и других документов.

Курсовой проект – результат самостоятельной работы студента по решению
поставленной небольшой задачи. Результаты работы оформляются в виде пояснительной
записки с приложением графиков, таблиц, чертежей, схем и других документов, а также
разработанного программного продукта.

Практика (учебная, производственная) является обязательной частью образовательной
программы, непосредственно ориентированной на профессионально-практическую подготовку
обучающихся. Аттестация по итогам практики осуществляется на основе отчета о проделанной
работе и его публичной защиты. Разделом учебной практики может являться научно-
исследовательская работа обучающегося. Таким образом, отчёт по практике может
рассматриваться либо как некоторый аналог курсовой работы, либо курсового проекта.
Результаты работы оформляются в виде пояснительной записки с приложением графиков,
таблиц, чертежей, схем и других документов, а также разработанного программного продукта.

\part{Общие требования к оформлению}

Настоящий документ устанавливает общие требования к структуре и правилам
оформления студенческих отчетов по итогам следующих видов деятельности: пояснительная
записка по дипломному проекту (работе), пояснительная записка по курсовому проекту
(работе), отчет по практике.

Пояснительная записка (ПЗ) выполняется на одной стороне листа белой бумаги
формата А4. Текст записки должен быть набран на компьютере с использованием текстового
процессора MS Word, OpenOffice Writer.

Окончательный вариант работы необходимо сброшюровать. Для дипломных и курсовых
проектов или работ обязателен твердый переплет.

Значения полей для различных видов документации приведены в таблице \ref{fields}.

% на таблице я сдался.
\begin{longtabu}{|X[5]|X[1]|X[1]|X[1]|X[1]|} \caption{Значения полей для различных видов документации}\\\hline
	\cch {Вид документации} & Левое & Верхнее & Правое & Нижнее \nr
	Курсовой проект, курсовая работа, отчёт по практике & 20 & 25 & 15 & 15 \nr
	ВКР бакалавров & 35 & 25 & 20 & 25 \nr
	ВКР магистров & 20 & 20 & 10 & 20 \label{fields}\nr
\end{longtabu}

Допускается представлять иллюстрации, таблицы и распечатки с компьютера на листах
формата А3 с соблюдением полей, указанных в таблице 1.1.

Для ВКР специалистов и бакалавров на каждом листе должен размещаться штамп учета
и хранения в соответствии с ГОСТ 19.602-78 и ГОСТ 19.601-78.

Следует форматировать текст ПЗ средствами текстового процессора. Недопустимо
использовать подряд идущие пробелы, пустые абзацы. Не рекомендуется использовать опцию
«Автоматическая расстановка переносов».

Текст документа должен быть отформатирован следующим образом:

\begin{easylist} \bullist
	@ межстрочный интервал – полуторный (1,5);
	@ шрифт – Times New Roman, размер – 12 пт, начертание – обычный, цвет – чёрный.
\end{easylist}

При выполнении ПЗ необходимо соблюдать равномерную плотность, контрастность и
четкость изображения по всему документу. В документе должны быть четкие, нерасплывшиеся
линии, буквы, цифры и знаки. Разрешается использовать компьютерные возможности
акцентирования внимания на определенных терминах, формулах, теоремах, применяя
различное начертание шрифта одной гарнитуры.

Опечатки и графические неточности, обнаруженные в процессе выполнения, допускается
исправлять подчисткой некачественно выполненной части текста (чертежа) или закрашиванием
белой краской и нанесением на том же листе исправлений чёрными чернилами или тушью.

Повреждения листов документов, помарки и следы не полностью удаленного прежнего
текста (графики) не допускаются.

\section{Структурные элементы текста пояснительной записки}

Структурными элементами текста пояснительной записки являются разделы,
подразделы, пункты, подпункты, абзацы и перечисления.

\subsection{Разделы}

{\bf Разделы} (первая ступень деления) должны иметь порядковые номера в пределах всей
пояснительной записки, обозначенные арабскими цифрами без точки и заголовки.

Для форматирования заголовка раздела следует применять стиль «Заголовок 1» со
следующими параметрами: Шрифт: Times New Roman, 12 пт, полужирный, выравнивание по
центру, отступ~--~нет, интервал после 18 пт, Уровень 1, многоуровневый.

Заголовки подразделов следует выполнять прописными буквами.

% TODO figure должен сам тянуть за собой абзац.
Примеры разделов в курсовых работах и проектах по дисциплинам, преподаваемым на
кафедре АСОИУ представлены на рисунке \ref{headers}.

% TODO \captionsetup[figure]{...}
\begin{figure}[h]
	\begin{framed}
		\setlength{\parindent}{1.25cm}
		\linespread{1.5}
		\centerline{\bf\uppercase{1 Технический проект}}
		\centerline{\bf\uppercase{2 Рабочий проект}}
		\centerline{\bf\uppercase{3 Программа и методика испытаний}}
	\end{framed}
	\caption{Заголовки разделов}
	\label{headers}
\end{figure}

Если заголовок состоит из нескольких предложений, то в конце каждого предложения
(кроме последнего) ставится точка. Если заголовок раздела занимает более одной строки, то
следует выставить междустрочный интервал – «Одинарный».

\subsection{Подразделы}

{\bf Подразделы} (вторая ступень деления) должны иметь нумерацию в пределах каждого
раздела. Номера подразделов состоят из номеров раздела и подраздела, разделенных точкой.
Номера подразделов обычно имеют заголовки, выполняемые строчными буквами (первая
буква – заглавная). Точки после номера подраздела и в конце заголовка не ставятся.

Для форматирования заголовка подраздела следует применять стиль «Заголовок 2» со
следующими параметрами: Шрифт: Times New Roman, 12 пт, полужирный, выравнивание по
левому краю, отступ первой строки 0,5 см, междустрочный интервал – одинарный, Уровень 2,
многоуровневый, отступы перед и после – 12 пт.

Примеры расположения заголовков подразделов и текста абзацев в курсовых работах и
проектах по дисциплинам, преподаваемым на кафедре АСОИУ \ref{subsects}.

\begin{figure}[h]
	\newcommand\ttt[1]{
		\vspace{6pt}
		\noindent
		\hspace* {.75cm}
		{\raggedright\bf #1}
		\vskip 6pt
	}
	\begin{framed}
		% TODO тут полуторный интервал просрался...
		\setlength{\parindent}{1.25cm}
		\linespread{1.5}
		\ttt{1.1 Описание предметной области}
		В этом подразделе необходимо привести все нужные понятия, определения, факты,
		математические (или другие) модели, позволяющие получить понятие о проектируемой
		системе или изучаемой области.
		
		\ttt{1.2 Технология обработки информации}
		Для научного исследования этот подраздел не обязателен, а в проектной работе –
		необходим. Следует привести алгоритмы, в соответствии с которыми входная информация
		преобразуется в выходную.
		
		\ttt{...}
		
		\ttt{3.4 Проверка работоспособности подсистемы «Демонстрация численного дифференцирования»}
		Так выглядит заголовок подраздела, если его длина более одной строки
	\end{framed}
	\caption{Заголовки подразделов}
	\label{subsects}
\end{figure}

\subsection{Пункт}

{\bf Пункт} -- часть раздела или подраздела, обозначенная номером, может иметь заголовок.
Номер пункта включает номера раздела, подраздела и порядковый номер пункта, разделенные
точкой. Выравнивание по левому краю. Пример – 1.1.1, 1.1.2, 1.1.3 и т.д.
Для форматирования заголовка пункта следует применять стиль «Заголовок 3» со
следующими параметрами: Шрифт: Times New Roman, 12 пт, полужирный, выравнивание по
левому краю, отступ первая строка 0,75 см, междустрочный интервал – одинарный, Уровень 3,
многоуровневый, отступы перед и после – 6 пт.

\subsection{Подпункт}

{\bf Подпункт} -- часть пункта, обозначенная номером, может иметь заголовок Номер
подпункта включает номер раздела, подраздела, пункта и порядковый номер подпункта,
разделенные точкой. Выравнивание по левому краю. Пример – 1.1.1.1, 1.1.1.2, 1.1.1.3 и т. д.
Для форматирования заголовка подпункта следует применять стиль «Заголовок 4» со
следующими параметрами: Шрифт: Times New Roman, 12 пт, полужирный, выравнивание по
левому краю, отступ первая строка 1,0 см, междустрочный интервал – одинарный, Уровень 4,
многоуровневый, отступы перед и после – 6 пт.
Если раздел или подраздел имеет только один пункт, или пункт имеет один подпункт, то
нумеровать его не следует.

\subsection{Абзац}

{\bf Абзац} -- логически выделенная часть текста, не имеющая номера. Абзац начинается с
красной строки – 1,25 см, выравнивание строки производится по ширине листа. Интервалы
перед и после абзаца равны нулю, междустрочный интервал равен 1,5.

\subsection{Перечисление}

Содержащиеся в тексте пункта или подпункта перечисления требований, указаний,
положений допускается нумеровать или маркировать. Каждый пункт перечисления следует
начинать с нового абзаца (отступы перед абзацем и после абзаца равны нулю, междустрочный
интервал 1,5).

Если текст пункта или подпункта перечисления не умещается в одну строку, то
следующие строки выравниваются по ширине. При форматировании перечисления следует
установить параметры списка: положение маркера (номера) отступ 1,25 см, положение текста
табуляция после 1,9 см, отступ 1,9 см (рис. 1.3).

\begin{figure}[h]
	\begin{framed}
		\centerline{TODO}
	\end{framed}
	\caption{Форматирование перечислений в MS Word}
	\label{subsects}
\end{figure}

Если пункты перечисления представляют собой части одного предложения, то каждый
пункт следует начинать строчной буквой и заканчивать точкой с запятой (за исключением
последнего пункта, который заканчивается точкой). При этом не рекомендуется использовать
нумерацию с точкой; её следует использовать, если каждый пункт перечисления представляет
собой отдельное предложение или несколько предложений. В этом случае каждый пункт
перечисления начинается с заглавной буквы и заканчивается точкой. И в том, и в другом случае
допускается использовать маркированный список и список со скобкой.

Следует использовать единый стиль форматирования списков для всего документа. В
случае, когда внутри перечислений нет дальнейшей детализации, в работе допускается один вид
нумерованного списка и один вид маркированного. Если внутри перечисления требуется
дальнейшая детализация, можно использовать многоуровневый список либо увеличить отступ

\begin{figure}[h]
	\begin{framed}
		\centerline{TODO, не я ебал easylist...}
	\end{framed}
	\caption{Вложенные перечисления}
	\label{subsects}
\end{figure}

Перечисления должны содержать как минимум два пункта. Не рекомендуется делать
ссылки на элементы перечисления.

\subsection{Таблицы}
Таблицы применяют для повышения наглядности и удобства сравнения показателей.
Недопустимо приводить таблицы, состоящие из одной строки (не считая головки) или
одного столбца. При переносе таблицы на другую страницу недопустимо отрывать номер
таблицы от самой таблицы или оставлять на странице только одну строку (не считая головки).

Таблицы нумеруют арабскими цифрами сквозной нумерацией (исключение - таблицы в
приложениях), точка в конце номера не ставится, например, «Таблица 2». Допускается
нумеровать таблицы в пределах раздела. В этом случае номер таблицы состоит из номера
раздела и порядкового номера таблицы, разделенных точкой, например, «Таблица 1.2». Таблицы
каждого приложения обозначают отдельной нумерацией арабскими цифрами с добавлением
перед цифрой обозначения приложения, например, «Таблица А.2», если она приведена в
приложении А, либо «Таблица 1.2», если нумерация приложений числовая. Если в документе
одна таблица, то она должна быть обозначена «Таблица 1» или «Таблица Б.1», если она
приведена в приложении Б. Таблицу, в зависимости от ее размера, помещают под текстом, в
котором впервые дана ссылка на нее или, при необходимости, в приложении к документу. На
все таблицы должны быть ссылки в документе. При ссылке следует писать слово «таблица» с
указанием ее номера, например, «...в таблице 2.1».

Таблицы оформляют в соответствии с рисунком 1.5 и размещают по центру страницы без
абзацного отступа. Таблица отделяется от основного текста отступами 6 пт.

\begin{figure}[h]
	\begin{framed}
		\centerline{TODO заюзать TikZ что ле...}
	\end{framed}
	\caption{Оформление таблицы}
	\label{subsects}
\end{figure}

Слово «Таблица» выравнивается по левому краюстраницы, интервал 1,5 отступы перед и
после 6 пт. После номера таблицы через тире указывается её название таблицы (первая буква
прописная, остальные строчные, точка в конце не ставится), без абзацного отступа. Название
таблицы должно отражать ее содержание, быть точным, кратким.

Заголовки граф и строк таблицы следует писать с прописной буквы, а подзаголовки
граф – со строчной буквы, если они составляют одно предложение с заголовком, или с
прописной буквы, если они имеют самостоятельное значение. В конце заголовков и
подзаголовков таблиц точки не ставят.

Таблицу с большим количеством строк допускается переносить на другой лист
(страницу). При переносе части таблицы на другой лист (страницу) слово «Таблица», ее номер
и название указывают один раз справа над первой частью таблицы. Над другими частями пишут
слово «Продолжение» и указывают номер таблицы, например – Продолжение таблицы 1.2, как
показано ниже. В каждой части повторяют головку таблицы.

\begin{longtabu}{|X[3]|X[1]|X[1]|} \caption{Структура таблицы}\\\hline
	% \phantom -- лучший друг верстальщика!!!
	\cch {\bf Название поля} & \cct {\bf Тип данных} & \cct {\bf Содержание } \nr
	\textbackslash phantom -- лучший друг верстальщика!!! & Ну типа... & Sample Text \nr
	TODO сделать что-то уже наконец с таблицами  & \phantom{a} & \phantom{a} \nr
	\phantom{a} & \phantom{a} & \phantom{a} \nr
	\phantom{a} & \phantom{a} & \phantom{a} \nr
	\phantom{a} & \phantom{a} & \phantom{a} \nr
	\phantom{a} & \phantom{a} & \phantom{a} \nr
	\phantom{a} & \phantom{a} & \phantom{a} \nr
	\phantom{a} & \phantom{a} & \phantom{a} \nr
	\phantom{a} & \phantom{a} & \phantom{a} \label{fields}\nr
\end{longtabu}

Заголовки, подзаголовки граф следует указывать в единственном числе. Заголовки граф,
как правило, записывают параллельно строкам таблицы. При необходимости допускается
перпендикулярное расположение заголовков граф. Разделять заголовки и подзаголовки боковика
и граф диагональными линиями не допускается.

Текст таблицы, заголовки и подзаголовки форматируются так же, как основной текст,
однако отступа первой строки нет. Заголовки и подзаголовки выравниваются по центру. Текст в
полях таблицы выравнивается по ширине, если поле достаточно широкое, в противном случае -
по левому краю. Числа в колонках таблицы рекомендуется выравнивать по правому краю так,
чтобы соответствующие разряды чисел во всей графе были точно один под другим. Исключение
составляют случаи, аналогичные приведённому в таблице \ref{struct}.

\begin{longtabu}{|X[1]|X[1]|} \caption{Структура таблицы}\\\hline
	\cch {\bf Название} & \cct {\bf Диапазон значений}\nr
	Количество специальностей & \centerline{30 -- 40} \nr 
	Количество групп & \centerline{3 - 5} \nr
	Количество дисциплин & \centerline{5 - 10} \nr
	Количество зачётов в семестре & \centerline{6 - 8} \label{struct}\nr
\end{longtabu}

Числовые значения величин в одной графе должны иметь, как правило, одинаковое
количество десятичных знаков. Дробные числа приводят в виде десятичных дробей.

Допускается текст, заголовки и подзаголовки таблиц выполнять через один интервал и
применять размер шрифта в таблице меньше, чем в тексте, но не менее 10 пт. Высота строк
таблицы должна быть не менее 8 мм.

Допускается помещать таблицу вдоль длинной стороны листа документа, если таблица
приведена в приложении.

Таблицу с большим количеством граф (графы таблицы выходят за формат страницы)
допускается делить на части и помещать одну часть под другой в пределах одной страницы. В
этом случае в каждой части таблицы повторяется боковик.

Таблицы с небольшим количеством граф допускается делить на части и помещать одну
часть рядом с другой на одной странице, при этом повторяют головку таблицы в соответствии с
рисунком. Рекомендуется разделять части таблицы двойной линией или линией удвоенной
толщины.

Таблицы слева, справа и снизу, как правило, ограничивают линиями. Головка таблицы
должна быть отделена линией от остальной части таблицы.

Если все показатели, приведенные в графах таблицы, выражены в одной и той же
единице измерения, то ее обозначение необходимо помещать над таблицей справа (после
заголовка). Если в большинстве граф таблицы приведены показатели, выраженные в одних и тех
же единицах измерения (например в рублях, миллиметрах, вольтах), но имеются графы с
показателями, выраженными в других единицах измерения, то над таблицей следует писать
наименование преобладающего показателя и обозначение его физической величины, например,
«Размеры в миллиметрах», «Напряжение в вольтах», а в подзаголовках остальных граф
приводить наименование показателей и (или) обозначения других единиц физических величин.

Если повторяющийся в разных строках графы таблицы текст состоит из одного слова, то
его после первого написания допускается заменять кавычками; если из двух и более слов, то
при первом повторении его заменяют словами «То же», а далее – кавычками
« «. Ставить кавычки вместо повторяющихся цифр, марок, знаков, математических и химических символов
не допускается. Если цифровые или иные данные в какой-либо строке таблицы не приводят, то в
ней ставят прочерк.

Графу «Номер по порядку» в таблицу включать не допускается. При необходимости
нумерации показателей, параметров или других данных порядковые номера следует указывать в
первой графе (боковике) таблицы непосредственно перед их наименованием.

Нумерация граф таблицы арабскими цифрами допускается, когда в тексте документа
имеются ссылки на них, при делении таблицы на части, а также при переносе части таблицы на
следующую страницу в соответствии с рисунком. При этом нумеруются арабскими цифрами
графы и (или) строки первой части таблицы.

Если параметры одной графы имеют одинаковые числовые значения в двух и более
последующих строках, то допускается этот параметр вписывать в таблицу для этих строк только
один раз.

При указании в таблицах последовательных интервалов значений величин,
охватывающих все значения ряда, перед ним пишут «от», «св.», «до»; в интервалах,
охватывающих любые значения величин, между величинами следует ставить тире. Пределы
размеров указывают от минимума к максимуму.

Если цифровые данные в графах таблицы выражены в различных единицах физических
величин, то их указывают в заголовке каждой колонки. Если все данные в строке приведены для
одной физической величины, то единицу физической величины указывают в колонке с
названием строки таблицы. Ограничительные слова «более», «не более», «менее», «не менее» и
др. должны быть помещены в одной строке или графе таблицы с наименованием
соответствующего показателя после обозначения его единицы физической величины, если
относятся ко всей строке или графе. При этом после наименования показателя перед
ограничительными словами ставится запятая.

Единицы измерения угловых величин (градусы, минуты, секунды) при отсутствии
горизонтальных линий указывают только в первой строке таблицы. При наличии в таблице
горизонтальных линий единицы измерения угловых величин проставляются во всех строках.

Для сокращения текста заголовков и подзаголовков колонок и строк отдельные понятия
можно заменять буквенными обозначениями, если они пояснены в тексте или приведены на
иллюстрациях, например: {\bf ХD} -- диаметр, {\bf H} --  высота, {\bf L} -- длина.

Показатели с одним и тем же буквенным обозначением группируют последовательно, в
порядке возрастания индексов, например: {\bf L}, {\bf L1},{\bf L2} и т.д.

\end{document}