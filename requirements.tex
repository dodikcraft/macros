\documentclass[a4paper,12pt]{article}

\usepackage[dvipdfm]{graphicx}
\usepackage[russian]{babel}
\usepackage{fontspec}
\usepackage[left=20mm,top=25mm,right=15mm,bottom=15mm,headsep=6pt,headheight=22pt]{geometry}
\usepackage{longtable, tabu, caption, setspace}
\usepackage{fancyhdr}
\usepackage[at]{easylist}

\pagestyle{fancy}

\setmainfont{Times New Roman}
\setmonofont{Go Mono}

\setlength{\parindent}{1.25cm}
\linespread{1.5}

% Списки

%% Список с буллетами
\def\bullist{
	\ListProperties(
	%Hide=100,
	Hang=true,
	Progressive*=.65cm,
	Margin=1.25cm,
	Align=fixed,
	%iq 3000 move: скрываем номера \фантом'ом, и вместо них пихаем буллеты
	Style*={\scshape$\bullet$\phantom},
	% АХАХАХАХ СУКА ЭТО СРАБОТАЛО
	% ну почти.. второй уровень косячит. для демо хватит.
	% TODO уровни после
	)
}

%% Нумерованный список
\def\numlist{
	\ListProperties(
	Hang=true,
	Progressive*=.65cm,
	Margin=1.25cm,
	Align=fixed,
}

% Колонтитулы
% TODO номер страницы должен отступать от основного текста
\newcommand*\sethdr[2] {
	\fancyhf{}
	\fancyhead[C]{
		\fontsize{12pt}{0pt}\selectfont
		\thepage
	}
	\fancyhead[L]{\vspace{12pt}\fontsize{10pt}{0pt}\selectfont\it #1}
	\fancyhead[C]{
		\fontsize{12pt}{0pt}\selectfont
		\thepage
	}
	\fancyhead[R]{\vspace{12pt}\fontsize{10pt}{0pt}\selectfont\it #2}
}

\renewcommand{\thepart}{\arabic{part}}

\newcommand*\unpart[1]{
	% TODO 12пт в начале
	\newpage
	\vspace{6pt}
	%\vspace как всегда ведет себя как долбоёб...
	%\vskip не лучше
	% короче ебаный латех не хочет мне вставлять 12 пунктов отступа сразу после page break
	\noindent
	\centerline{\bf\uppercase{#1}}
	\vskip 12pt
	\addcontentsline{toc}{part}{#1}
}

% TODO мб звёздочка лишняя?
\renewcommand*\part[1]{
	\newpage
	\addtocounter{part}{1}
    	\addcontentsline{toc}{part}{#1}
	\vspace{6pt}
	\noindent
	\centerline{\bf\uppercase{{\thepart}\ #1}}
	\vskip 12pt
}

\renewcommand*\section[1]{
	\addtocounter{section}{1}
	\addcontentsline{toc}{section}{#1}
	\vspace{12pt}
	\noindent
	\hspace* {.5cm}
	{\raggedright\bf\thepart.\thesection\ #1}
	\vskip 12pt
}

\renewcommand*\subsection[1]{
	\addtocounter{subsection}{1}
	\addcontentsline{toc}{paragraph}{#1}
	\vspace{6pt}
	\noindent
	\hspace* {.75cm}
	{\raggedright\bf\thepart.\thesubsection\ #1}
	\vskip 6pt
}

% Настройка таблиц
\tabulinesep=_12pt
\renewcommand{\thetable}{\thepart.\arabic{table}}
\captionsetup[table]{singlelinecheck=false,justification=raggedright,position=top,%
	format=plain,font={rm,onehalfspacing},labelsep=endash,indention=0pt,%
	skip=-6pt
	}
\def\cch{\multicolumn{1}{|c|}}
\def\cct{\multicolumn{1}{c|}}
\def\nr{\\\hline}


\usepackage{framed}

\sethdr{Кафедра АСОИУ}{Требования к оформлению студенческих работ}

\begin{document}
\unpart{Введение}

В настоящее время получение высшего профессионального образования подразумевает
не просто получение некоторого набора знаний, но целого комплекса компетенций – знаний,
умений, навыков, направленных на решение профессиональных задач. Одной из компетенций
является умение оформлять научно-технические отчёты по результатам выполненной работы.
На формирование этой компетенции направлено выполнение курсовых проектов и работ,
отчётов по практике.

Выпускная квалификационная работа является заключительным этапом обучения
студента на соответствующей ступени высшего профессионального образования и имеет своей
целью:
\begin{easylist}\NewList\bullist
	@ систематизацию, закрепление и расширение теоретических и практических знаний по
	соответствующей специальности (направлению подготовки), а также формирование
	навыков применения этих знаний при решении конкретных научных, научно-
	технических, экономических, социально-культурных и производственных задач;
	@ развитие навыков ведения самостоятельной работы и овладения методикой
	теоретических, экспериментальных и научно-практических исследований,
	используемых при выполнении выпускной квалификационной работы;
	@ приобретение опыта систематизации полученных результатов исследований,
	формулировку новых выводов и положений как результатов выполненной работы, а
	также опыта их публичной защиты.
\end{easylist}

Выпускная квалификационная работа (ВКР) в области информатики и информационных
технологий представляет собой законченную разработку в профессиональной области, где:

\begin{easylist} \bullist
	@ сформулирована актуальность и место решаемой задачи информационного обеспечения 
	в предметной области;
	@ анализируется литература и информация, полученная с помощью глобальных сетей
	по функционированию подобных систем в данной области или в смежных
	предметных областях;
	@ определяются и конкретно описываются выбранные выпускником объемы, методы и
	средства решаемой задачи, иллюстрируемые данными и формами выходных
	документов, используемых при реализации поставленной задачи информационного
	обеспечения;
	@ анализируются предлагаемые пути, способы, а также оценивается экономическая,
	техническая и (или) социальная эффективность их внедрения в информационную
	среду в области применения.
\end{easylist}

ВКР бакалавра может быть представлена в виде дипломной работы, либо в виде
дипломного проекта. ВКР магистра представляет собой диссертацию, то есть научную работу,
при её выполнении оформлении следует придерживаться требований ГОСТ 7.32-2001.

Дипломная работа –-- самостоятельная комплексная работа студента, главной целью и
содержанием которой является всесторонний анализ или научные исследования по одному из
вопросов теоретического или практического характера по профилю специализации. Дипломная
работа представляет собой теоретическое или экспериментальное исследование одной из
актуальных научных проблем. Результаты работы оформляются в виде пояснительной записки с
приложением графиков, таблиц, чертежей, схем и других документов.

Дипломный проект –-- это ВКР, которая содержит решение поставленной задачи,
оформленное в виде конструкторских, технологических, программных и других проектных
документов. Главной целью и содержанием дипломного проекта являются разработка
проектного решения, связанного с созданием или совершенствованием информационной
системы на базе использования современных информационных технологий, средств
вычислительной техники и передачи данных, экономико-математических методов и моделей,
разработка технологических процессов обработки информации и решение организационных
вопросов управления производством. Результаты работы оформляются в виде пояснительной
записки с приложением графиков, таблиц, чертежей, схем и других документов, а также
разработанного программного продукта.

Курсовая работа представляет собой небольшое по объёму научное исследование.
Результаты работы оформляются в виде пояснительной записки с приложением графиков,
таблиц, чертежей, схем и других документов.

Курсовой проект – результат самостоятельной работы студента по решению
поставленной небольшой задачи. Результаты работы оформляются в виде пояснительной
записки с приложением графиков, таблиц, чертежей, схем и других документов, а также
разработанного программного продукта.

Практика (учебная, производственная) является обязательной частью образовательной
программы, непосредственно ориентированной на профессионально-практическую подготовку
обучающихся. Аттестация по итогам практики осуществляется на основе отчета о проделанной
работе и его публичной защиты. Разделом учебной практики может являться научно-
исследовательская работа обучающегося. Таким образом, отчёт по практике может
рассматриваться либо как некоторый аналог курсовой работы, либо курсового проекта.
Результаты работы оформляются в виде пояснительной записки с приложением графиков,
таблиц, чертежей, схем и других документов, а также разработанного программного продукта.

\part{Общие требования к оформлению}

Настоящий документ устанавливает общие требования к структуре и правилам
оформления студенческих отчетов по итогам следующих видов деятельности: пояснительная
записка по дипломному проекту (работе), пояснительная записка по курсовому проекту
(работе), отчет по практике.

Пояснительная записка (ПЗ) выполняется на одной стороне листа белой бумаги
формата А4. Текст записки должен быть набран на компьютере с использованием текстового
процессора MS Word, OpenOffice Writer.

Окончательный вариант работы необходимо сброшюровать. Для дипломных и курсовых
проектов или работ обязателен твердый переплет.

Значения полей для различных видов документации приведены в таблице \ref{fields}.

% на таблице я сдался.
\begin{longtabu}{|X[5]|X[1]|X[1]|X[1]|X[1]|} \caption{Значения полей для различных видов документации}\\\hline
	\cch {Вид документации} & Левое & Верхнее & Правое & Нижнее \nr
	Курсовой проект, курсовая работа, отчёт по практике & 20 & 25 & 15 & 15 \nr
	ВКР бакалавров & 35 & 25 & 20 & 25 \nr
	ВКР магистров & 20 & 20 & 10 & 20 \label{fields}\nr
\end{longtabu}

Допускается представлять иллюстрации, таблицы и распечатки с компьютера на листах
формата А3 с соблюдением полей, указанных в таблице 1.1.

Для ВКР специалистов и бакалавров на каждом листе должен размещаться штамп учета
и хранения в соответствии с ГОСТ 19.602-78 и ГОСТ 19.601-78.

Следует форматировать текст ПЗ средствами текстового процессора. Недопустимо
использовать подряд идущие пробелы, пустые абзацы. Не рекомендуется использовать опцию
«Автоматическая расстановка переносов».

Текст документа должен быть отформатирован следующим образом:

\begin{easylist} \bullist
	@ межстрочный интервал – полуторный (1,5);
	@ шрифт – Times New Roman, размер – 12 пт, начертание – обычный, цвет – чёрный.
\end{easylist}

При выполнении ПЗ необходимо соблюдать равномерную плотность, контрастность и
четкость изображения по всему документу. В документе должны быть четкие, нерасплывшиеся
линии, буквы, цифры и знаки. Разрешается использовать компьютерные возможности
акцентирования внимания на определенных терминах, формулах, теоремах, применяя
различное начертание шрифта одной гарнитуры.

Опечатки и графические неточности, обнаруженные в процессе выполнения, допускается
исправлять подчисткой некачественно выполненной части текста (чертежа) или закрашиванием
белой краской и нанесением на том же листе исправлений чёрными чернилами или тушью.

Повреждения листов документов, помарки и следы не полностью удаленного прежнего
текста (графики) не допускаются.

\section{Структурные элементы текста пояснительной записки}

Структурными элементами текста пояснительной записки являются разделы,
подразделы, пункты, подпункты, абзацы и перечисления.

\subsection{Разделы}

Разделы (первая ступень деления) должны иметь порядковые номера в пределах всей
пояснительной записки, обозначенные арабскими цифрами без точки и заголовки.

Для форматирования заголовка раздела следует применять стиль «Заголовок 1» со
следующими параметрами: Шрифт: Times New Roman, 12 пт, полужирный, выравнивание по
центру, отступ~--~нет, интервал после 18 пт, Уровень 1, многоуровневый.

Заголовки подразделов следует выполнять прописными буквами.

% TODO figure должен сам тянуть за собой абзац.
Примеры разделов в курсовых работах и проектах по дисциплинам, преподаваемым на
кафедре АСОИУ \ref{headers}.

% TODO \captionsetup[figure]{...}
\begin{figure}[h]
	\begin{framed}
		\centerline{\bf\uppercase{1 Технический проект}}
		\centerline{\bf\uppercase{2 Рабочий проект}}
		\centerline{\bf\uppercase{3 Программа и методика испытаний}}
	\end{framed}
	\caption{Заголовки разделов}
	\label{headers}
\end{figure}

\end{document}