\documentclass[a4paper,12pt]{article}

\usepackage[dvipdfm]{graphicx}
\usepackage[russian]{babel}
\usepackage{fontspec}
\usepackage[left=20mm,top=25mm,right=15mm,bottom=15mm,headsep=6pt,headheight=22pt]{geometry}
\usepackage{multirow}
\usepackage{longtable}
\usepackage{tabu}
\usepackage{caption}
\usepackage{setspace}
\usepackage{fancyhdr}
\usepackage[at]{easylist}

\pagestyle{fancy}

\setmainfont{Times New Roman}
\setmonofont{Go Mono}

\setlength{\parindent}{1.25cm}
\renewcommand{\baselinestretch}{1.5}
\renewcommand{\smallskip}{\vskip{6pt}}
\renewcommand{\medskip}{\vskip{6pt}}
\renewcommand{\bigskip}{\vskip{12pt}}
%\setlength{\skip}{0}
\frenchspacing

\def\nohyphen{
	\pretolerance=10000
	\tolerance=2000 
	\emergencystretch=10pt
}

% Списки

%% Список с буллетами
\def\bullist{
	\ListProperties(
	%Hide=100,
	Hang=true,
	Progressive*=.65cm,
	Margin=1.25cm,
	Align=fixed,
	%iq 3000 move: скрываем номера \фантом'ом, и вместо них пихаем буллеты
	Style*={\scshape$\bullet$\phantom},
	% АХАХАХАХ СУКА ЭТО СРАБОТАЛО
	% ну почти.. второй уровень косячит. для демо хватит.
	% TODO уровни после
	)
}

%% Нумерованный список
\def\numlist{
	\ListProperties(
	Hang=true,
	Progressive*=.65cm,
	Margin=1.25cm,
	Align=fixed,
}

% Колонтитулы
% TODO номер страницы должен отступать от основного текста
\newcommand*\sethdr[2] {
	\fancyhf{}
	\fancyhead[C]{
		\fontsize{12pt}{0pt}\selectfont
		\thepage
	}
	\fancyhead[L]{\vspace{12pt}\fontsize{10pt}{0pt}\selectfont\it #1}
	\fancyhead[C]{
		\fontsize{12pt}{0pt}\selectfont
		\thepage
	}
	\fancyhead[R]{\vspace{12pt}\fontsize{10pt}{0pt}\selectfont\it #2}
}

\renewcommand{\thepart}{\arabic{part}}

\newcommand*\unpart[1]{
	% TODO 12пт в начале
	\newpage
	\vspace{6pt}
	%\vspace как всегда ведет себя как долбоёб...
	%\vskip не лучше
	% короче ебаный латех не хочет мне вставлять 12 пунктов отступа сразу после page break
	\noindent
	\centerline{\bf\uppercase{#1}}
	\vskip 12pt
	\addcontentsline{toc}{part}{#1}
}

% TODO мб звёздочка лишняя?
\renewcommand*\part[1]{
	\newpage
	\addtocounter{part}{1}
    	\addcontentsline{toc}{part}{#1}
	\vspace{6pt}
	\noindent
	\centerline{\bf\uppercase{{\thepart}\ #1}}
	\vskip 12pt
}

\renewcommand*\section[1]{
	{
		\nohyphen
		\linespread{1}
		\addtocounter{section}{1}
		\addcontentsline{toc}{section}{#1}
		\vspace{12pt}
		\noindent
		\hspace* {.5cm}
		{\raggedright\bf\thepart.\thesection\ #1}
		\vskip 12pt
	}
}

\renewcommand*\subsection[1]{
	{
		\nohyphen
		\linespread{1}
		\addtocounter{subsection}{1}
		\addcontentsline{toc}{paragraph}{#1}
		\vspace{6pt}
		\noindent
		\hspace* {.75cm}
		{\raggedright\bf\thepart.\thesubsection\ #1}
		\vskip 6pt
	}
}

% Настройка таблиц
\tabulinesep=_6pt
\renewcommand{\thetable}{\thepart.\arabic{table}}
\captionsetup[table]{singlelinecheck=false,justification=raggedright,position=top,%
	format=plain,font={rm,onehalfspacing},labelsep=endash,indention=0pt,%
	skip=6pt,belowskip=0pt
	}
\def\cch{\multicolumn{1}{|c|}}
\def\cct{\multicolumn{1}{c|}}
\def\nr{\\\hline}
\newcommand*\autohead[2]{
	#1\endfirsthead
	\caption*{Продолжение таблицы \ref{#2}}\nr
	#1
	\endhead
}

% Иллюстрации
\DeclareCaptionLabelFormat{myfig}{Рисунок #2}
\captionsetup[figure]{singlelinecheck=false,justification=centering,position=bottom,%
	format=plain,font={rm,onehalfspacing},labelformat=myfig,labelsep=endash,indention=0pt,%
	skip=6pt,belowskip=0pt
	}
\renewcommand{\thefigure}{\thepart.\arabic{figure}}

\addto{\captionsrussian}{
	\renewcommand*{\contentsname}{
		\noindent
		\centerline{\bf\uppercase{Содержание}}
		\vskip 12pt
	}
}

\setlength{\belowcaptionskip}{-10pt}
\setlength{\textfloatsep}{6pt}
